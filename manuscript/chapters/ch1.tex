% Chapter 1

\chapter{Introduction}

\label{ch1}

%----------------------------------------------------------------------------------------

% Define some commands to keep the formatting separated from the content 
\newcommand{\keyword}[1]{\textbf{#1}}
\newcommand{\tabhead}[1]{\textbf{#1}}
\newcommand{\code}[1]{\texttt{#1}}
\newcommand{\file}[1]{\texttt{\bfseries#1}}
\newcommand{\option}[1]{\texttt{\itshape#1}}

%----------------------------------------------------------------------------------------

Graphene’s unique and excellent electronic properties, as well as its small volume,
make it an ideal candidate for its use in electronic applications [** kretz]

One reason that graphene research has progressed so fast is that the laboratory procedures enabling us to obtain high-quality graphene are relatively simple and cheap. Many graphene characteristics measured in experiments have exceeded those obtained in any other material, with some reaching theoretically predicted limits: room-temperature electron mobility of 2.5 * 10 5 cm V-1 s-1 (ref. 3) (theoretical limit $\sim$ 2 * 10 5 cm2 V-1 s-1); a Young's modulus of 1 TPa and intrinsic strength of 130 GPa (ref. 5, very close to that predicted by theory6); very high thermal conductivity (above 3000 W mK-1; ref. 7); optical absorption of exactly pi*alpha $\simeq$ 2.3 (in the infrared limit, where alpha is the fine structure constant)8; complete impermeability to any gases9, ability to sustain extremely high densities of electric current (a million times higher than copper)10. Another property of graphene, already demonstrated11,12,13, is that it can be readily chemically functionalized. [**roadmap novoselov]

\section{Thesis outline}



