% Chapter 1

\chapter{Introduction}

\label{ch1}

%----------------------------------------------------------------------------------------

% Define some commands to keep the formatting separated from the content 
\newcommand{\keyword}[1]{\textbf{#1}}
\newcommand{\tabhead}[1]{\textbf{#1}}
\newcommand{\code}[1]{\texttt{#1}}
\newcommand{\file}[1]{\texttt{\bfseries#1}}
\newcommand{\option}[1]{\texttt{\itshape#1}}

%----------------------------------------------------------------------------------------

Since Geim and Novoselov first isolated and charaterized of graphene\parencite{Geim2007} the study of its properties and applications has become an active and still growing field of research especially within condensed matter physics and materials science\parencite{Houtsma2021}. Graphene’s exceptional electronic properties, combined with its small volume, make it an ideal candidate for its use in electronic applications, yet its gapless, semimetallic nature prevents it from being used as a logic device with distinguishable ON and OFF states\parencite{Allen2010}. One way to do this is via electron confinement in one direction, in the form of graphene nanoribbons (GNRs).\\


**GNR to NPG: general reasons**\\
%After the emergence of bottom-up on-surface synthesis of organic covalent structures [1,2] Specifically designed organic molecules are deposited on metallic substrates where they self-assemble and, by increasing temperature, they bind through a surface-catalyzed reaction via specific functional groups, forming highly robust covalent nano-structures. [3] Over the years, a number of different types of carbon nanomaterials have been fabricated in this way, such as extended 2D covalent organic frameworks, [2,4,5] 1D and circular molecular oligomers, [3,6–8] 1D organic polymers, [9–11] and nanographenes. [12–15] Among all these, the so-called graphene nanoribbons [16,17] (GNRs), first reported via on-surface synthesis in 2010, [18] have received enormous attention as potential platforms for future carbon nanoelectronics. [19–22] This is partly because of their fully \(\pi\)-conjugated structure allowing for efficient electron transport, plus their confined dimensionality equipping them with an electronic band gap which is essential for logic-gate applications. In this direction, electron transport through GNRs has been measured multiple times in solid-state devices [23–26] and also on metallic surfaces using the tip of a scanning tunneling microscope (STM).\\


**Molecular switches: what for, applications?**\\
%Use of external stimuli (e.g. light, temperature, pH) to controllably induce molecular level structural and/or chemical changes can be exploited for using individual molecules like switches in electronic devices.1–3 Unlike the top-down fabrication of inorganic siliconbased transistors, molecular switches are synthesised via bottom-up chemical design. As such, there are many different examples of switchable molecular systems based on a wide range of chemistries.4 Since electric field (E-field) driven semiconductor-based switches are key components in electrical circuitry, in recent years extensive research efforts have been devoted to fabricating junctions of single-molecules or selfassembled monolayers whose conductance can be reversibly switched by means of external E-fields. Such E-field induced molecular switching can be achieved by means of changes in the redox state of electrochemically active molecules,5–7 molecular conformational rearrangements,


Relate molecular switches and sensing\\

**sensing: motivation and problems. NPG: array of sensors?\\
general properties of sensors (meng) + specific example (shekirev), not "accurate" enough(?) -> atomically precise array of sensors (talk about molecular switch sensors first)\\
%
** our proposal...**\\

\section{Thesis outline}

In chapter 2 we carefully describe the methods used for calculations in this thesis, namely density functional theory (DFT) through the program SIESTA, and the \textsc{sisl} Python package. Chapter 3 is a review of the state of the art regarding NPG structures and transport calculations within them, and we reproduce some results found in the literature. In chapter 4 our main results and calculations are presented, and chapter 5 summarizes our work and makes the concluding statements, as well as a discussion regarding future work.


