% Chapter 1

\chapter{Theoretical and computational tools} % Main chapter title
% Theory and programs?

\label{Chapter1} % For referencing the chapter elsewhere, use \ref{Chapter1} 

%----------------------------------------------------------------------------------------

% Define some commands to keep the formatting separated from the content 
\newcommand{\keyword}[1]{\textbf{#1}}
\newcommand{\tabhead}[1]{\textbf{#1}}
\newcommand{\code}[1]{\texttt{#1}}
\newcommand{\file}[1]{\texttt{\bfseries#1}}
\newcommand{\option}[1]{\texttt{\itshape#1}}

%----------------------------------------------------------------------------------------
\section{DFT with SIESTA}
Goal is to solve many-body problem through Schrödinger's equation.
\subsection{Basics of DFT}
\subsection{SIESTA}
summary: In this method the effect of
the core electrons is described by soft norm-conserving
pseudopotentials34 and the electronic structure of the valence
electrons is expanded in a basis set of numerical atomic orbitals with finite range
%----------------------------------------------------------------------------------------

\section{Transport through NEGF}

In this section the theory of nonequilibrium Green's functions (NEGF) will be presented, as the underlying theory of various programs that will be used (\textsc{TranSIESTA} *** and \textsc{TBrans}***) to study transport properties of the desired materials. Knowledge of (first- and second-quantized) quantum mechanics  will be assumed, and the starting point will be an equilibrium state, which will lay the foundations and the basic properties of Green's functions. We will then delve into NEGF through the Keldysh formalism, and afterwards explore a simple and conceptually useful reformulation of this theory by S. Datta \textit{et al}***. Finally, we will overview the implementation in \textsc{TranSIESTA}.


\subsection{Equilibrium Green's functions}


\subsection{Keldysh formalism}
\subsection{TranSIESTA and TBtrans}
\subsection{sisl}


