% Chapter 1

\chapter{Theoretical and computational tools} % Main chapter title
% Theory and programs?

\label{Chapter1} % For referencing the chapter elsewhere, use \ref{Chapter1} 

%----------------------------------------------------------------------------------------

% Define some commands to keep the formatting separated from the content 
\newcommand{\keyword}[1]{\textbf{#1}}
\newcommand{\tabhead}[1]{\textbf{#1}}
\newcommand{\code}[1]{\texttt{#1}}
\newcommand{\file}[1]{\texttt{\bfseries#1}}
\newcommand{\option}[1]{\texttt{\itshape#1}}

%----------------------------------------------------------------------------------------
\section{Density Functional Theory}

Density-functional theory (DFT) is an approach to study the electronic structure of many-body problems, which allows the computational treatment large and complex systems. In fact, one of the reasons why DFT has become an essential tool in many areas of physics including condensed-matter theory is the increasing availability and power of computational processing. DFT is mainly based on the fact that any property of a system of many interacting particles can be viewed as a functional of the ground state
density [**martin]. The famous paper by Hohenberg and Kohn in 1964 [** in martin] laid the groundwork of modern DFT, while the formulation presented in a 1965 paper by Kohn and Sham [** in martin] has prevailed as one of the most useful approaches up to this day. In the following subsections we will present the basics of this method, and then review a specific implementation, namely SIESTA, which will be used throughout this work.


\subsection{Basics of DFT}
The main problem to be solved is the many-body time-independent Schrödinger's equation, in the spin-unpolarized case:
\begin{equation}
\bm H \ket{\Psi(\{\bm r\},\{\bm R\})} = E \ket{\Psi(\{\bm r\},\{\bm R\})},
\end{equation}

where \(\{\bm r\}\) and \(\{\bm R\}\) are the electron and ion position vectors, respectively. The most basic Hamiltonian contains a kinetic term plus a potential energy term describing two particle interactions: electron-ion, electron-electron and ion-ion [** martin]:
\begin{equation}
\begin{split}
\bm{H} &= \bm T_{\mathrm e} + \bm V_{\mathrm{en}}+\bm V_{\mathrm{ee}}+\bm T_n+\bm V_{\mathrm{nn}}\\
&= -\frac{\hbar^2}{2m_{\mathrm{e}}}\sum_i\nabla_i^2-\sum_{i,I}\frac{Z_Ie^2}{\abs{\bm r_i-\bm R_i}} + \frac 1 2 \sum_{i\neq j}\frac{e^2}{\abs{\bm r_i-\bm r_j}}\\
&\quad-\sum_I\frac{\hbar^2}{2M_I}\nabla_I^2+\frac 1 2 \sum_{I\neq J}\frac{Z_IZ_Je^2}{\abs{\bm R_I-\bm R_J}},
\end{split}
\end{equation}

which after the well-known Born-Oppenheimer approximation [** martin], whereby electrons are adiabatically separated from ions owing to their large mass difference, can be replaced with
\begin{equation}
\bm{H}\equiv\bm H_{\mathrm e} = \bm T_{\mathrm e} + \bm V_{\mathrm{en}}+\bm V_{\mathrm{ee}},
\end{equation}


so that the corresponding electron eigenfunctions only depend on \(\{\bm R\}\) parametrically. As the name implies, DFT is based upon electron density rather than the explicit wavefunctions, and it does so through two theorems first proved by Hohenberg and Kohn [** martin], which we will now describe. The ground-state wavefunction satisfies\footnote{Note that we write \(\bm V_{\mathrm{ext}}\) instead of the electron-ion potential for generality, but the former usually includes the latter.}
\begin{equation}
\bm H \ket{\Psi_0}=(\bm T_{\mathrm e} + \bm V_{\mathrm{ext}}+\bm V_{\mathrm{ee}})\ket{\Psi_0}=E_0\ket{\Psi_0},
\end{equation}

so along with the ground-state energy it is determined by the external potential Therefore, so is the density, since the wavefunction is a unique functional of this quantity by the relation [** Solyom]

\begin{equation}
	n_{\mathrm{e}}(\bm{r})=N_{\mathrm{e}} \int \mathrm{d} \bm{r}_{2} \cdots \int \mathrm{d} \bm{r}_{N_{\mathrm{e}}}\left|\Psi\left(\bm{r}, \bm{r}_{2}, \ldots, \bm{r}_{N_{\mathrm{e}}}\right)\right|^{2}
\end{equation}

where \(N_{\mathrm{e}}\) is the total number of electrons. The first Hohenberg-Kohn theorem states that this relationship between the external potential and the density is bilateral.

\begin{theorem}
The ground-state wavefunction \(\ket{\Psi_0}\), and hence the ground-state expectation values of any observable, is a unique functional of the ground-state density \(n_{\mathrm e}(\bm{r})\).
\end{theorem}

Thus, \(n_{\mathrm e}(\bm r)\) univocally determines \(\bm V_{\mathrm{ext}}\), and so we can write the following functional relation
\begin{equation}
	E\left[n_{\mathrm{e}}(\bm{r})\right]=F\left[n_{\mathrm{e}}(\bm{r})\right]+\int V_{\mathrm{ext}}(\bm{r}) n_{\mathrm{e}}(\bm{r}) \mathrm{d} \bm{r},
\end{equation}

where \(F\) is the Hohenberg-Kohn functional defined by [** papi]
\begin{equation}
\begin{split}
F\left[n_{\mathrm{e}}(\bm{r})\right]&=\expval{\bm T_{\mathrm e}+\bm V_{\mathrm{ee}}}{\Psi_0}=\bm T_{\mathrm e}\left[n_{\mathrm{e}}(\bm{r})\right]+\bm E_{\mathrm{ee}}\left[n_{\mathrm{e}}(\bm{r})\right]\\
&=\bm T_{\mathrm e}\left[n_{\mathrm{e}}(\bm{r})\right]+\bm E_{\mathrm{H}}\left[n_{\mathrm{e}}(\bm{r})\right]+\bm E_{\mathrm{Q}}\left[n_{\mathrm{e}}(\bm{r})\right],
\end{split}
\end{equation}

where we detach from the electron-electron interaction energy the Hartree term, which describes the classical self-interaction of the electron density:
\begin{equation}
E_{\mathrm{H}}\left[n_{\mathrm{e}}(\bm{r})\right]=\frac {e^2} 2 \int\D\bm r\D\bm{r}'\frac{n_{\mathrm{e}}(\bm{r})n_{\mathrm{e}}(\bm{r}')}{\abs{\bm{r} - \bm{r}'}},
\end{equation}

and express the rest as a non-classical part. %bla bla bla 2nd theorem

\subsection{SIESTA}
summary: In this method the effect of
the core electrons is described by soft norm-conserving
pseudopotentials and the electronic structure of the valence
electrons is expanded in a basis set of numerical atomic orbitals with finite range
%----------------------------------------------------------------------------------------

%\section{Transport through NEGF}
%
%In this section the theory of nonequilibrium Green's functions (NEGF) will be presented, as the underlying theory of various programs that will be used (\textsc{TranSIESTA} *** and \textsc{TBrans}***) to study transport properties of the desired materials. Knowledge of (first- and second-quantized) quantum mechanics  will be assumed, and the starting point will be an equilibrium state, which will lay the foundations and the basic properties of Green's functions. We will then delve into NEGF through the Keldysh formalism, and afterwards explore a simple and conceptually useful reformulation of this theory by S. Datta \textit{et al}***. Finally, we will overview the implementation in \textsc{TranSIESTA}.
%
%
%\subsection{Equilibrium Green's functions}
%
%
%\subsection{Keldysh formalism}
%\subsection{TranSIESTA and TBtrans}
%\subsection{sisl}


